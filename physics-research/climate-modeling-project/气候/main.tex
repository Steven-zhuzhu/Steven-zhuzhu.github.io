\documentclass[11pt]{article}

\usepackage{amsmath,amssymb}
\usepackage{graphicx}
\usepackage{geometry}
\usepackage{natbib}
\usepackage{subcaption}
\usepackage{placeins}
\usepackage{float}
\usepackage{hyperref}
\usepackage{graphicx}
\usepackage{float}
\usepackage{subcaption}

\setlength{\textfloatsep}{10pt}
\setlength{\floatsep}{8pt}
\setlength{\intextsep}{10pt}
\setlength{\abovecaptionskip}{4pt}
\setlength{\belowcaptionskip}{0pt}

\renewcommand{\topfraction}{0.9}
\renewcommand{\bottomfraction}{0.9}
\renewcommand{\textfraction}{0.05}
\renewcommand{\floatpagefraction}{0.8}

\geometry{margin=1in}

\title{Energy Balance Climate Modelling with Radiative Forcing and Deep-Ocean Heat Uptake}
\author{Wuhua Zhu\\BASIS INTERNATIONAL SCHOOL PARK LANE HARBOUR \\
Guangzhou, China\\wuhua\_zhu@hotmail.com}



\begin{document}
\maketitle
\thispagestyle{empty}
\newpage

\begin{abstract}
This study uses an energy balance model (EBM) to simulate changes in the global mean surface air temperature. The model incorporates radiative forcing from greenhouse gases, atmospheric aerosols, and volcanic aerosols, and is extended to include heat exchange between the Earth’s surface and the deep ocean in order to represent the ocean’s role as a heat reservoir.

Simulations were performed using two model configurations: a baseline model without deep ocean heat uptake and an extended model that includes deep ocean dynamics. The simulated surface temperature anomalies were compared with observational data to assess model performance. The inclusion of deep ocean heat uptake improves agreement between simulated and observed temperature anomalies. Using mean absolute error (MAE) as a performance metric, the extended model produces lower error than the baseline configuration.

The model is further used to examine the effects of changes in radiative forcing parameters and climate feedback strength on temperature response. In addition, the physical implications and limitations of several global cooling strategies are discussed within the simplified modeling framework.

Overall, the results highlight the importance of deep ocean heat exchange in regulating surface temperature changes and demonstrate that relatively simple energy balance models can capture key features of the global climate response to external forcing.
\end{abstract}
\newpage


\section{Introduction}
In the summer of 2021, my mother's hometown suffered a once-in-a-century flood, with daily rainfall almost reaching the local annual average. The town was submerged, and the post-disaster scene was shocking. In August 2025, I personally witnessed a super typhoon in Sanya destroying the tranquil island in an instant. These two extreme weather events made me deeply aware of the ferocity of natural forces and the imminent threat of climate change, and also made me determined to use my knowledge to understand and even attempt to predict these phenomena.

Recent extreme weather events have highlighted the direct impacts of strong storms on populated regions and raised public interest in understanding climate drivers. In late September 2025, Super Typhoon Ragasa—a powerful tropical cyclone in the Northwest Pacific—approached southern China and made landfall in Guangdong Province, bringing heavy winds, intense rainfall, and extensive disruptions to transportation and daily life in the Pearl River Delta and surrounding areas \cite{Ragasa2025}. Winds exceeded 140\,kph near landfall and authorities issued the highest tropical cyclone warnings as coastal cities faced flood risks and storm surge warnings. Such events motivate further inquiry into the mechanisms that influence climate variability and extreme weather patterns.

Earth’s climate is largely affected by the balance between energy received from the Sun and energy
reemitted into space. This balance is captured by an energy balance model, which incorporates the
effects of radiative forcing on Earth’s surface temperature. Changes in radiative forcing arise from both natural and anthropogenic sources and have been shown to significantly alter climate dynamics \cite{Shindell2013RF}.

Earth reaches radiative equilibrium and a corresponding planetary equilibrium temperature when net
radiative forcing is zero. This equilibrium can be disrupted by greenhouse gases, which allow short-wave solar radiation to pass through Earth’s atmosphere while inhibiting the escape of long-wave radiation emitted by the Earth’s surface \cite{Wild2013EnergyBalance}. This trapped radiation produces positive radiative forcing and leads to an increase in global surface temperature. Conversely, most atmospheric aerosols produce negative radiative forcing by reflecting incoming solar radiation back into space, thereby cooling the Earth’s surface \cite{Charlson1992Aerosols}.

In addition to radiative forcing, Earth’s climate is strongly influenced by the ocean’s ability to absorb and store heat. The deep ocean, in particular, absorbs a substantial fraction of excess heat due to its large heat capacity, which exceeds that of the atmosphere, land surface, and upper ocean combined \cite{Geoffroy2013EBM}. This heat uptake acts to moderate the rate of surface temperature increase \cite{Menzel2019OceanHeat}.

This study models changes in Earth’s global mean surface air temperature by incorporating radiative
forcing from greenhouse gases, atmospheric aerosols, and heat exchange with the deep ocean.


\section{Model Formulation}

The model considered in this study is an energy balance model designed to describe Earth's
global mean surface temperature. This class of models is based on the fundamental physical
principle that changes in Earth's internal energy are equal to the difference between energy
received by and energy emitted from the Earth's surface. Mathematically, this relationship
can be expressed as

\begin{equation}
\text{Heat content} = \text{Heat capacity} \times \text{Temperature} = C\,T ,
\label{eq:heat_content}
\end{equation}

where $C$ denotes the effective heat capacity of the climate system and $T$ represents the
global mean surface temperature.

Assuming that Earth's heat capacity $C$ is constant and differentiating
Eq.~\eqref{eq:heat_content} with respect to time $t$, a preliminary energy balance equation
is obtained as

\begin{equation}
C \frac{dT}{dt} = E_{\text{in}} - E_{\text{out}} ,
\label{eq:energy_balance}
\end{equation}

where $E_{\text{in}}$ and $E_{\text{out}}$ represent the average incoming and outgoing energy
fluxes per unit surface area of the Earth, respectively.

The final form of the model is derived by decomposing the energy fluxes into specific physical
processes that separately influence the energy received or retained by the Earth and the
energy emitted from the system. The incoming energy term is expressed as

\begin{equation}
E_{\text{in}} = E_{\text{sol}} + E_{\text{gas}} + E_{\text{vol}} + E_{\text{aer}} ,
\label{eq:Ein}
\end{equation}

while the outgoing energy term is given by

\begin{equation}
E_{\text{out}} = G(T) + E_{\text{ocean}} .
\label{eq:Eout}
\end{equation}

Here, $E_{\text{sol}}$ represents incoming solar radiation, $E_{\text{gas}}$ accounts for
radiative forcing due to greenhouse gases, $E_{\text{vol}}$ and $E_{\text{aer}}$ represent
radiative effects associated with volcanic and anthropogenic aerosols, respectively.
The function $G(T)$ describes outgoing longwave radiation as a function of temperature, and
$E_{\text{ocean}}$ represents heat exchange between the surface and the deep ocean.
\subsection{Solar Radiation ($E_{\text{sol}}$)}

The incoming solar energy received by the Earth, denoted by $E_{\text{sol}}$, is estimated by
approximating the Earth as a sphere and assuming that solar radiation is emitted in parallel
rays. Under these assumptions, the effective cross-sectional area intercepting solar radiation
is circular, as illustrated in Fig.~\ref{fig:solar_model}.

\begin{figure}[ht]
\centering
\includegraphics[width=0.8\textwidth]{solar_radiation_model.png}
\caption{Schematic illustration of the simplified solar radiation model.}
\label{fig:solar_model}
\end{figure}


Under this geometric approximation, the total solar energy intercepted by the Earth is given by
$S_0 \pi R_e^2$, where $S_0$ denotes the solar constant and $R_e$ is the radius of the Earth.
This energy is distributed over the entire surface area of the Earth, $4\pi R_e^2$.
Consequently, the average incoming solar radiation per unit surface area is

\begin{equation}
\frac{S_0 \pi R_e^2}{4\pi R_e^2} = \frac{S_0}{4}.
\end{equation}

A fraction of the incoming radiation is reflected back into space due to the Earth's albedo,
denoted by $\alpha$. Accounting for this effect, the net solar radiation absorbed by the Earth
is expressed as

\begin{equation}
E_{\text{sol}} = \frac{(1-\alpha) S_0}{4},
\label{eq:Esol}
\end{equation}

where $\alpha \approx 0.3$ and $S_0 \approx 1368~\mathrm{W\,m^{-2}}$.
\subsection{Greenhouse Effects ($E_{\text{gas}}$)}

Greenhouse gases, in particular carbon dioxide (CO$_2$), methane (CH$_4$), and nitrous oxide
(N$_2$O), partially inhibit the escape of longwave radiation emitted from the Earth's surface
into space. This trapped radiation leads to an increase in surface temperature and is commonly
represented in climate models through radiative forcing terms.

In this model, the radiative forcing contributions of the three major greenhouse gases are
parameterized using empirical relationships reported in the literature. The corresponding
radiative forcing equations and associated constants are summarized in Table~\ref{tab:ghg_rf}.

\begin{table}[ht]
\centering
\caption{Radiative forcing parameterizations for CO$_2$, CH$_4$, and N$_2$O,
adapted from \cite{Boucher2002}.}
\label{tab:ghg_rf}
\begin{tabular}{lll}
\hline
Gas & Radiative forcing equation & Constants \\
\hline
CO$_2$ &
$\Delta F_{\text{CO}_2} = \alpha \ln \!\left( \dfrac{C}{C_0} \right)
+ \beta \left( \sqrt{C} - \sqrt{C_0} \right)$
&
$\alpha = 4.841,\; \beta = 0.0906$
\\[6pt]

CH$_4$ &
$\Delta F_{\text{CH}_4}
= \alpha \left( \sqrt{M} - \sqrt{M_0} \right)
- f(M,N_0) - f(M_0,N_0)$
&
$\alpha = 0.036$
\\[6pt]

N$_2$O &
$\Delta F_{\text{N}_2\text{O}}
= \alpha \left( \sqrt{N} - \sqrt{N_0} \right)
- f(M_0,N) - f(M_0,N_0)$
&
$\alpha = 0.12$
\\
\hline
\end{tabular}
\end{table}


Here, $C$, $M$, and $N$ denote the atmospheric concentrations of CO$_2$, CH$_4$, and N$_2$O,
respectively, while $C_0$, $M_0$, and $N_0$ represent their corresponding pre-industrial
reference concentrations. The overlap function $f(M,N)$ accounts for spectral interactions
between methane and nitrous oxide and is defined following established formulations in
radiative forcing studies.

The total additional energy retained by the Earth due to greenhouse gases is then expressed
as the sum of the individual radiative forcing contributions,

\begin{equation}
E_{\text{gas}} = \Delta F_{\text{CO}_2}
+ \Delta F_{\text{CH}_4}
+ \Delta F_{\text{N}_2\text{O}} .
\label{eq:Egas}
\end{equation}
\subsection{Volcanic Aerosols ($E_{\text{vol}}$)}

Aerosols emitted during volcanic eruptions reduce the amount of solar radiation reaching the
Earth's surface by reflecting incoming radiation back into space. This process leads to a
temporary cooling effect and is incorporated into the model through the following system of
equations:

\begin{equation}
\frac{d\,\mathrm{OT}^{\text{vol}}}{dt}
= E_{\text{vol}}^{\text{src}} - \frac{1}{\tau^{\text{vol}}}\,\mathrm{OT}^{\text{vol}},
\label{eq:volcanic_ot}
\end{equation}

\begin{equation}
E_{\text{vol}} = c^{\text{vol}} \, \mathrm{OT}^{\text{vol}},
\label{eq:Evol}
\end{equation}

where $\mathrm{OT}^{\text{vol}}$ denotes the optical thickness of volcanic aerosols,
$E_{\text{vol}}^{\text{src}}$ represents the aerosol injection rate, $\tau^{\text{vol}}$ is
the characteristic decay timescale, and $E_{\text{vol}}$ ($\mathrm{W\,m^{-2}}$) denotes the
radiative forcing contribution associated with volcanic aerosols.

\subsection{Anthropogenic Aerosols ($E_{\text{aer}}$)}

Anthropogenic aerosols, predominantly sulphur dioxide emissions, exert a radiative forcing
effect similar to that of volcanic aerosols by reflecting solar radiation back into space.
The contribution of human-caused aerosols to the Earth's energy balance is parameterized as

\begin{equation}
E_{\text{aer}} = c^{\text{aer}} \bigl( E^{\text{SO}_2}(t) - E^{\text{SO}_2}(0) \bigr),
\label{eq:Eaer}
\end{equation}

where $E^{\text{SO}_2}(t)$ ($10^9\,\mathrm{kg\,yr^{-1}}$) denotes the mass of sulphur dioxide
emitted at time $t$, $E^{\text{SO}_2}(0)$ is the corresponding pre-industrial emission level,
and $c^{\text{aer}}$ is a dimensionless scaling coefficient.

\subsection{Outgoing Longwave Radiation}

In typical energy balance models, the Earth is approximated as a grey body that emits longwave
radiation according to the Stefan--Boltzmann law,

\begin{equation}
j^* = \sigma \varepsilon T^4,
\label{eq:stefan_boltzmann}
\end{equation}

where $\sigma = 5.6703 \times 10^{-8}\,\mathrm{W\,m^{-2}\,K^{-4}}$ is the Stefan--Boltzmann
constant, $\varepsilon$ is the effective emissivity of the Earth ($0 < \varepsilon < 1$),
and $T$ denotes the global mean surface temperature.

Empirical observations indicate that the blackbody approximation alone does not fully capture
the Earth's outgoing radiation response. Instead, a linear approximation of the outgoing
radiative flux is commonly adopted \cite{KaperEngler2013MathClimate}. This approximation is obtained by
performing a first-order Taylor expansion of $G(T)$ about the pre-industrial equilibrium
temperature $T_0 = 14^{\circ}\mathrm{C}$:

\begin{equation}
G(T) \approx G(T_0) + G'(T_0)(T - T_0)
= -A - B T,
\label{eq:linearized_G}
\end{equation}

where $B < 0$ is the climate feedback parameter ensuring model stability and $A$
($\mathrm{W\,m^{-2}}$) is a constant determined by imposing pre-industrial energy balance.

Under pre-industrial equilibrium conditions, the net energy balance satisfies

\begin{equation}
E_{\text{in}} = E_{\text{out}},
\label{eq:preindustrial_balance}
\end{equation}

which yields

\begin{equation}
A = \frac{S_0(1-\alpha)}{4} + B T_0 \approx 221.2~\mathrm{W\,m^{-2}}.
\label{eq:A_constant}
\end{equation}
A substantial fraction of excess energy in the Earth's climate system is absorbed and stored
by the deep ocean, typically at depths greater than 500~m. Owing to its large heat capacity,
the deep ocean acts as a long-term heat reservoir and moderates the rate of increase in surface
temperature. This heat exchange between the surface and deep ocean is incorporated into the
model through the following parameterization:

\begin{equation}
E_{\text{ocean}} = \gamma \bigl( T_{\text{deep}} - T_{\text{surface}} \bigr),
\label{eq:Eocean}
\end{equation}

where $T_{\text{surface}}$ denotes the surface ocean temperature (at depths shallower than
500~m), $T_{\text{deep}}$ represents the temperature of the deep ocean, and $\gamma$
($\mathrm{W\,m^{-2}\,K^{-1}}$) is an empirically derived coupling coefficient that governs the
rate of heat exchange between the surface and deep ocean layers.

\subsection{Final Model}

Combining the parameterizations for incoming solar radiation, greenhouse gas forcing,
aerosol effects, outgoing longwave radiation, and deep ocean heat uptake, the final
energy balance model governing the evolution of the global mean surface air temperature
is expressed as

\begin{equation}
C \frac{dT}{dt}
=
\frac{S_0 (1-\alpha)}{4}
- (A - B T)
- E_{\text{ocean}}
+ \Delta F_{\text{CO}_2}
+ \Delta F_{\text{CH}_4}
+ \Delta F_{\text{N}_2\text{O}}
+ E_{\text{vol}}
+ E_{\text{aer}} ,
\label{eq:final_model}
\end{equation}

where $C$ denotes the effective heat capacity of the climate system and $T$ represents the
global mean surface air temperature. Descriptions, units, and physical interpretations of
all model variables and parameters are provided in Appendix~B.


\section{Data and Methods}

The energy balance model described in Section~2 was implemented as a numerical simulation
using Python in a Jupyter notebook environment. The simulation was executed on the Google
Colab platform, and the source code is provided in Appendix~D. Model inputs consist of
time series data for atmospheric greenhouse gas concentrations, volcanic and anthropogenic
aerosol emissions, and observed global mean surface air temperature used for model validation.

Atmospheric concentrations of carbon dioxide (CO$_2$), methane (CH$_4$), and nitrous oxide
(N$_2$O) were obtained from two primary sources. Recent observational estimates covering the
period from 1979 to 2019 were sourced from the National Oceanic and Atmospheric Administration
(NOAA). Earlier estimates extending back to year 1~AD were obtained from ice
core records collected at Law Dome, Antarctica. These datasets provide
annual mean concentrations of CO$_2$ (ppm) and CH$_4$ and N$_2$O (ppb), which were used as
inputs for the radiative forcing calculations.

Data for sulphur dioxide (SO$_2$) emissions and volcanic aerosol loading were obtained from
the Carbonator Climate Model. This dataset provides annual values of visible aerosol optical
depth at a wavelength of 550~nm, as well as global SO$_2$ emissions reported on 5- and
10-year intervals. Annual SO$_2$ emission levels were estimated by linear interpolation
between consecutive reported values. The resulting time series is illustrated in
Figure~\ref{fig:ghg_concentrations}.
\begin{figure}[H]
    \centering

    \begin{subfigure}{0.32\textwidth}
        \centering
        \includegraphics[width=\textwidth]{ch4.png}
        \caption{CH$_4$ concentration}
        \label{fig:ch4_conc}
    \end{subfigure}
    \hfill
    \begin{subfigure}{0.32\textwidth}
        \centering
        \includegraphics[width=\textwidth]{n2o.png}
        \caption{N$_2$O concentration}
        \label{fig:n2o_conc}
    \end{subfigure}
    \hfill
    \begin{subfigure}{0.32\textwidth}
        \centering
        \includegraphics[width=\textwidth]{co2.png}
        \caption{CO$_2$ concentration}
        \label{fig:co2_conc}
    \end{subfigure}

    \caption{Annual atmospheric concentrations of CO$_2$, CH$_4$, and N$_2$O.}
    \label{fig:ghg_concentrations}
\end{figure}



Model validation was performed using observational surface temperature data from NASA's
Land--Ocean Temperature Index (LOTI). This dataset reports annual global mean
temperature anomalies relative to the pre-industrial baseline period (1850). To obtain
absolute surface air temperatures, a constant offset of 14.15~$^{\circ}$C was added to the
reported anomalies. The resulting temperature series was compared with simulated model
outputs to assess model performance.

\section{Results}

Initial simulations were conducted using an energy balance model that did not incorporate heat exchange between the surface and the deep ocean. An extended version of the model, which includes deep ocean heat uptake, was then evaluated to assess the impact of this process on model performance.

Simulated and observed global mean surface temperature anomalies were compared for the two model configurations. Temperature anomalies are defined relative to the pre-industrial baseline (1850), with an absolute reference temperature of 14.15~$^\circ$C used to compute annual simulated and observed temperatures.

Model performance was quantified using the mean absolute error (MAE) between simulated and observed temperature anomalies. For the baseline model excluding deep ocean heat uptake, the MAE was found to be 0.167~$^\circ$C. In contrast, the extended model incorporating deep ocean dynamics achieved a reduced MAE of 0.154~$^\circ$C.

These results demonstrate that inclusion of deep ocean heat exchange improves agreement between simulated and observed surface temperature anomalies.




\section{Model Evaluation}

This section presents a qualitative evaluation and physical interpretation of the model
behaviour, focusing on how different radiative forcings influence the mean global surface
temperature, the timescale over which the system responds to these forcings, and the transfer
of energy between the surface and the deep ocean. The implications of these behaviours for
different global cooling strategies are also examined.

A series of diagnostic plots is used to analyse the response of the model to changes in
radiative forcing from CO$_2$, CH$_4$, N$_2$O, SO$_2$, and volcanic aerosols, alongside the
corresponding surface and deep ocean temperature anomalies, as shown in Figure~\ref{fig:c1}.




Sensitivity
analyses are performed by varying forcing coefficients to explore both short- and long-term
temperature responses and the conditions under which the model approaches energy balance.


Since aerosols produce negative radiative forcing, their forcing coefficients are negative, and
reducing their magnitude leads to a decrease in temperature. It can be observed that decreasing
the forcing coefficient associated with volcanic emissions primarily affects temperature in the
short term and has a limited impact on long-term temperature trends, as shown in Figure~A2(b).
In contrast, decreasing the forcing coefficient associated with SO$_2$ results in a reduction
in temperature over the long term, as shown in Figure~A2(d). When all forcing coefficients are
set to zero, the model reaches energy balance and a constant temperature is obtained, as shown
in Figure~A2(e) and Figure~A2(f).



The time required for surface temperature to respond to changes in radiative forcing also
depends strongly on the heat capacity of the surface and deep ocean, as well as the rate at
which heat is transferred into the ocean and emitted into space. Another plotting feature in
the Jupyter notebook is used to demonstrate this behaviour. It visualises surface and deep
ocean temperature anomalies alongside the effective temperature anomaly resulting from
incoming solar energy and outgoing blackbody radiation. Interactive sliders allow the heat
capacity of the surface and deep ocean to be adjusted, as well as the rate of heat transfer
between the ocean and the atmosphere.

The heat capacity of a reservoir determines the amount of heat required to produce a change
in temperature. Specifically, the surface heat capacity affects how sensitive the surface
temperature is to changes in radiative forcing. This effect can be highlighted by increasing
the albedo to simulate a planet that reflects a large proportion of incoming solar radiation
(for instance, an ice-covered planet). This reduces the constant incoming heat energy, such
that perturbations due to radiative forcing imbalances can be accentuated. When the surface
heat capacity is reduced, greater dips in surface temperature are observed following volcanic
eruptions, demonstrating the high sensitivity of surface temperature to changes in radiative
forcing, as shown in Figure~A3(b). A similar, though slightly delayed, effect is observed in
the deep ocean temperature when the deep ocean heat capacity is also reduced, as shown in
Figure~A3(c).




The climate feedback parameter represents a form of negative feedback and, within the
compartmental model, describes the rate at which heat energy is transferred to the sink.
As a result, the climate feedback parameter determines how efficiently the Earth can
naturally cool by emitting heat energy into space. Increasing the absolute value of this
parameter leads to a more stable system in which the overall temperature of both the surface
and deep ocean is less affected by positive radiative forcing, thereby slowing global warming.
In this case, temperature increases occur more gradually and remain below 2$^{\circ}$C
until approximately the year 2600, as shown in Figure~A4(a). Conversely, decreasing the
absolute value of the climate feedback parameter reduces system stability, accelerating the
effects of global warming and causing the surface temperature to rise above 2$^{\circ}$C by
the year 2017, as shown in Figure~A4(b).




\section{Implications and Limitations}
The goal of the Paris Agreement is to ensure that the rise in mean global temperature remains
below 1.5$^{\circ}$C above pre-industrial levels \cite{UN2015Paris}. Model projections
indicate that if anthropogenic greenhouse gas emissions continue to increase following the
observed cubic trends shown in Figure~A5(a), (b), and (c), the global mean surface temperature
will exceed the 1.5$^{\circ}$C threshold by approximately the year 2040, as shown in
Figure~A1(b).

CO$_2$ is well known to have the strongest impact on climate among anthropogenic greenhouse
gases. Figure~A1(b) shows that CO$_2$ produces the greatest positive radiative forcing and thus
exerts a larger influence on rising temperature than CH$_4$ and N$_2$O. However, the model
fails to accurately simulate the effects of reducing future CO$_2$ emissions, as it does not
explicitly consider the carbon cycle. Incorporating the carbon cycle would allow the absorption
and release of carbon and CO$_2$ by the upper and deep ocean, as well as uptake by vegetation
and soils, followed by the delayed release of CO$_2$ back into the atmosphere. This would make
it possible to simulate delayed radiative forcing associated with dormant CO$_2$ and carbon
stored within the Earth system.

As a result, the model produces an unrealistic projection in which surface temperatures would
cease to rise if net-zero greenhouse gas emissions were achieved in 2021, maintaining a stable
equilibrium temperature approximately 1$^{\circ}$C above the pre-industrial level. More
advanced climate models demonstrate that temperatures would continue to increase for some time
after net-zero emissions are reached, before stabilising at a new equilibrium. This equilibrium
would depend primarily on atmospheric CO$_2$ concentrations and the amount of carbon stored in
the ocean and terrestrial reservoirs at the time net-zero emissions are achieved. This suggests
that realistic global cooling strategies would require the removal of CO$_2$ from the
atmosphere to prevent temperatures from rising beyond 1.5$^{\circ}$C.

The IPCC targets the removal of 100--1000~Gt of CO$_2$ within this century in order to limit
warming to a 1.5$^{\circ}$C equilibrium temperature \cite{NRC2015ClimateIntervention}. Selecting
500~Gt as a representative target, a rough estimate indicates that this would correspond to
an atmospheric CO$_2$ concentration of approximately 316.8~ppm by the year 2100. Further
details of this calculation are provided in Appendix~C.

By projecting atmospheric CO$_2$ concentrations to decrease linearly from 414~ppm in 2021
to 316.8~ppm in 2100, the model simulates a temperature increase that remains well below that
envisioned by the IPCC, as shown in Figure~3. However, the model cannot accurately assess
global cooling strategies that involve direct extraction of CO$_2$ from the atmosphere,
such as bioenergy with carbon capture and storage (BECCS) and direct air capture, as it does
not represent the persistence of CO$_2$ and carbon within the climate system. As a result,
the model would significantly underestimate future temperature increases following reductions
in atmospheric CO$_2$ levels.

\begin{figure}[H]
  \centering
  \includegraphics[
    width=\linewidth,
    height=0.6\textheight,
    keepaspectratio
  ]{d2.png}
  \caption{}
  \label{fig:d2}
\end{figure}


Solar geoengineering refers to a class of climate engineering approaches aimed at reducing
the amount of solar energy entering the Earth's climate system. One such method, known as
Marine Cloud Brightening (MCB), involves increasing cloud reflectivity over the oceans,
thereby increasing the planetary albedo. Increasing the albedo linearly from 0.31 to 0.35
between 2021 and 2100 simulates MCB counteracting the radiative forcing associated with cubic
growth in CO$_2$ emissions, as shown in Figure~4. However, this approach cannot be extended
indefinitely, as albedo is a bounded ratio with an upper limit of 1. MCB may therefore serve
as a rapidly effective and relatively inexpensive short-term cooling strategy \cite{HillMing2010MCB},
while longer-term solutions must focus on reversing the greenhouse effect and removing
greenhouse gases from the atmosphere.

\begin{figure}[H]
  \centering
  \includegraphics[
    width=\linewidth,
    height=0.32\textheight,
    keepaspectratio
  ]{d3.png}
  \caption{}
  \label{fig:d3}
\end{figure}



Stratospheric Aerosol Injection (SAI) represents another solar geoengineering approach, in
which aerosols (such as sulphur dioxide) are released into the stratosphere to induce negative
radiative forcing. Considering only the radiative forcing associated with CO$_2$ and SO$_2$,
Figure~5 illustrates the SO$_2$ emission rates required to offset the positive radiative
forcing produced by cubic growth in CO$_2$ emissions. However, studies suggest that SAI may
lead to a range of harmful side effects, including changes in rainfall distribution and
impacts on water resources in certain regions. MCB may exhibit similar environmental risks
associated with increased sulphur emissions, although it is generally considered a safer
and more viable alternative.

\begin{figure}[H]
  \centering
  \includegraphics[
    width=\linewidth,
    height=0.32\textheight,
    keepaspectratio
  ]{d4.png}
  \caption{}
  \label{fig:d4}
\end{figure}

Other global cooling strategies involve enhancing ocean heat uptake. Increasing ocean
alkalinity by dispersing pulverised carbonate minerals onto the sea surface is one approach
that can increase the ocean's capacity to absorb heat. Another proposed
strategy involves restoring coastal blue carbon ecosystems, such as mangroves and seagrasses,
which have the potential to remove substantial amounts of carbon from the atmosphere. These cooling strategies can be partially simulated within the model
by increasing the rate of heat transfer between the surface and deep ocean. However, enhanced
ocean heat uptake leads to increased ocean temperatures and, consequently, sea level rise.
Increasing the ocean feedback parameter from 0.73 to 1.5 and the deep ocean heat capacity from
109 to 120 results in a reduction of the projected surface temperature in 2050 from 1.7$^{\circ}$C
to 1.4$^{\circ}$C, as shown in Figure~6. These results indicate that enhancing ocean heat
uptake may reduce surface temperatures, but must be applied with caution due to its potential
to accelerate sea level rise.

\begin{figure}[H]
    \centering
    \includegraphics[width=0.95\textwidth]{d5.png}
    \caption{}
    \label{fig:d5}
\end{figure}


\section{Conclusion}
This study presents a simplified energy balance model for simulating changes in global mean surface air temperature. The model accounts for radiative forcing from greenhouse gases, atmospheric aerosols, and volcanic activity, and includes an extension that represents heat exchange between the surface and the deep ocean.

Comparisons between model configurations show that incorporating deep ocean heat uptake leads to improved agreement with observed surface temperature anomalies. Quantitative evaluation using mean absolute error indicates that the extended model performs better than the baseline model without deep ocean dynamics. This result delivers the role of the ocean in moderating surface temperature changes over time.

The model is also used to explore how variations in radiative forcing and climate feedback parameters influence temperature response. While the simplified framework limits the ability to capture regional effects and complex feedbacks, it provides a clear and interpretable representation of the dominant physical processes governing global temperature change.

Overall, this study demonstrates that relatively simple energy balance models can offer useful insight into the relationship between external forcing, ocean heat uptake, and global temperature evolution. Such models provide a valuable foundation for understanding more complex climate models and for interpreting the physical implications of climate mitigation strategies.
\clearpage

\bibliographystyle{plainnat}
\bibliography{references}



\clearpage


\appendix

\renewcommand{\thefigure}{A\arabic{figure}}
\setcounter{figure}{0}

\section{Additional Figures}

\begin{figure}[htbp]
    \centering
    \includegraphics[width=0.8\textwidth]{c1.png}
    \caption{Radiative forcing imbalances and corresponding surface temperature anomalies.}
    \label{fig:c1}
\end{figure}

\begin{figure}[htbp]
    \centering
    \includegraphics[width=\textwidth]{c2.png}
    \caption{
    Simulated radiative forcing imbalances and corresponding surface temperature responses.
    Panels (a), (c), and (e) show changes in radiative forcing due to volcanic aerosols,
    SO$_2$, and all forcings set to zero, respectively.
    Panels (b), (d), and (f) show the resulting surface temperature anomalies under the
    corresponding forcing configurations.
    }
    \label{fig:radiative_forcing_sensitivity}
\end{figure}

\begin{figure}[htbp]
    \centering
    \includegraphics[width=0.95\textwidth]{c3.png}
    \caption{Surface and deep ocean temperature anomalies under different model parameter configurations. Panels (a)--(c) show simulations using empirically determined parameters, modified surface heat capacity, and increased albedo, respectively.}
    \label{fig:c3}
\end{figure}

\begin{figure}[htbp]
    \centering
    \includegraphics[width=0.95\textwidth]{c4.png}
    \caption{Simulated surface temperature anomalies under different values of the climate feedback parameter. Panel (a) shows the response when the feedback parameter is reduced from $-1.3$ to $-10$, while panel (b) shows the response when it is reduced from $-1.3$ to $-0.1$.}
    \label{fig:c4}
\end{figure}

\begin{figure}[htbp]
    \centering
    \includegraphics[width=0.95\textwidth]{d1.png}
    \caption{}
    \label{fig:d1}
\end{figure}


\clearpage
\section{Additional Table}
\renewcommand{\thetable}{B\arabic{table}}
\setcounter{table}{0}

\begin{table}[htbp]
\centering
\caption{Description, units and dimensions of model variables}
\label{tab:variables}

\begin{tabular}{llll}
\hline
Variable & Description & Unit & Dimension \\
\hline
$T$ & Global annual surface temperature & $^\circ$C & $\theta$ \\
$E_{sol}$ & Solar energy received by Earth & W m$^{-2}$ & M T$^{-3}$ \\
$\Delta F_{CO_2}$ & Radiative forcing of carbon dioxide & W m$^{-2}$ & M T$^{-3}$ \\
$C_{CO_2}$ & Mass of CO$_2$ & $10^9$ kg & M \\
$F_{CO_2}(t)$ & Mass of CO$_2$ emitted per unit time & $10^9$ kg t$^{-1}$ & M T$^{-1}$ \\
$\tau_{CO_2}$ & Half-life of CO$_2$ in atmosphere & t$^{-1}$ & T$^{-1}$ \\
$\Delta F_{CH_4}$ & Radiative forcing of methane & W m$^{-2}$ & M T$^{-3}$ \\
$C_{CH_4}$ & Mass of CH$_4$ & $10^9$ kg & M \\
$F_{CH_4}(t)$ & Mass of CH$_4$ emitted per unit time & $10^9$ kg t$^{-1}$ & M T$^{-1}$ \\
$\tau_{CH_4}$ & Half-life of CH$_4$ in atmosphere & t$^{-1}$ & T$^{-1}$ \\
$\Delta F_{N_2O}$ & Radiative forcing of nitrous oxide & W m$^{-2}$ & M T$^{-3}$ \\
$C_{N_2O}$ & Mass of N$_2$O & $10^9$ kg & M \\
$F_{N_2O}(t)$ & Mass of N$_2$O emitted per unit time & $10^9$ kg t$^{-1}$ & M T$^{-1}$ \\
$\tau_{N_2O}$ & Half-life of N$_2$O in atmosphere & t$^{-1}$ & T$^{-1}$ \\
$OT^{vol}$ & Optical thickness due to volcanic aerosols & m & L \\
$E_{vol}$ & Energy change related to volcanic aerosols & W m$^{-2}$ & M T$^{-3}$ \\
$E_{SO_2}(t)$ & Mass of human aerosol emitted per unit time & $10^9$ kg t$^{-1}$ & M T$^{-1}$ \\
$T_{deep-ocean}$ & Temperature of deep ocean & $^\circ$C & $\theta$ \\
$T_{surface-ocean}$ & Temperature of surface ocean & $^\circ$C & $\theta$ \\
\hline
\end{tabular}
\end{table}

\clearpage

\begin{table}[htbp]
\centering
\caption{Description, units and dimensions of model parameters}
\label{tab:parameters}

\begin{tabular}{llll}
\hline
Parameter & Description & Unit & Dimension \\
\hline
$C$ & Heat capacity of the Earth & W m$^{-2}$ $^\circ$C & M T$^{-3}\theta^{-1}$ \\
$\alpha$ & Planetary albedo & -- & -- \\
$S_0$ & Solar constant & W m$^{-2}$ & M T$^{-3}$ \\
$k_{CO_2}$ & CO$_2$ emission constant & $10^9$ kg t$^{-3}$ & M T$^{-3}$ \\
$k_{CH_4}$ & CH$_4$ emission constant & $10^9$ kg t$^{-3}$ & M T$^{-3}$ \\
$k_{N_2O}$ & N$_2$O emission constant & $10^9$ kg t$^{-3}$ & M T$^{-3}$ \\
$c^{vol}$ & Volcanic aerosol emission constant & W m$^{-3}$ & M T$^{-3}$ L$^{-1}$ \\
$c^{SO_2}$ & Human aerosol emission constant & W m$^{-2}$ $10^{-9}$ kg$^{-1}$ t$^{-1}$ & T$^{-2}$ \\
$A$ & Linearised outgoing radiation constant & W m$^{-2}$ & M T$^{-3}$ \\
$B$ & Climate feedback parameter & W m$^{-2}$ $^\circ$C & M T$^{-3}\theta^{-1}$ \\
$\gamma$ & Ocean heat uptake coefficient & W m$^{-2}$ $^\circ$C & M T$^{-3}\theta^{-1}$ \\
\hline
\end{tabular}
\end{table}

\clearpage
\section{Calculation}

The following calculation was used to estimate the mass of carbon dioxide (CO$_2$) in the atmosphere in 2021, given that the recorded atmospheric concentration was 414~ppm.

The concentration of a solute in parts per million is defined as
\begin{equation*}
\text{concentration}_{\text{ppm}} = 
\frac{\text{mass of solute}_{\text{Gt}}}{\text{mass of solution}_{\text{Gt}}}
\times 10^{6}.
\end{equation*}

Since CO$_2$ constitutes less than 1\% of the total atmospheric mass, the mass of the atmosphere (the solution) was assumed to be constant at $5.148 \times 10^{6}$~Gt. The mass of atmospheric CO$_2$ in 2021 is therefore approximated as
\begin{equation*}
\text{mass of solute} \approx 414 \times 5.148 = 2131.72~\text{Gt}.
\end{equation*}

Reducing this value by the target removal amount of 500~Gt yields an estimated remaining atmospheric CO$_2$ mass of
\begin{equation*}
\frac{2131.72 - 500}{5.148} = 316.8~\text{ppm}.
\end{equation*}

This concentration corresponds to the target level required for the global mean surface temperature increase to peak at approximately 1.5$^\circ$C by the year 2100.

\clearpage
\section{Code Source}
The complete source code used for all models, figures, and simulations is available as a Google Colab notebook at the following link:
\href{https://colab.research.google.com/drive/1-nZWwWlL7LP2gufw53lPwOfqHAHtMM0D?usp=sharing}{HERE}.


\end{document}

